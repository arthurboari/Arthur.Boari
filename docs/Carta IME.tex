\documentclass[12pt,a4paper]{article}
\usepackage[utf8]{inputenc}
\usepackage[T1]{fontenc}
\usepackage[brazilian]{babel}
\usepackage{lipsum}
\usepackage{fancyhdr}
\usepackage{color}

%links
\usepackage{hyperref}
\hypersetup{
	colorlinks=true,
	linkcolor=black,
	citecolor=black,
	filecolor=black,
	urlcolor=blue
}

%cabeçalho
\fancyhf{} % limpa os cabecalhos e rodapés
\fancyhead[L]{ARTHUR BOARI} % define o cabeçalho personalizado
\fancyhead[R]{\thepage / \pageref{LastPage}}
\pagestyle{fancy} % sem definir esse comando, o cabeçalho personalizado não é exibido

\hyphenation{hardware software Li-nux am-bien-te diag-nos-ti-car coor-de-na-ção 
	FAE-PE Recovery TelEduc Williams UFLA }

\begin{document}
	
	\begin{flushright}
		\href{http://lattes.cnpq.br/7113151378838886}{Arthur Boari}\\
		Rua Ametista, 317 – Parque das Pedras Preciosas\\
		Lavras, Minas Gerais\\
		\href{mailto:eng.arthurboari@gmail.com}{eng.arthurboari@gmail.com}
	\end{flushright}


	\begin{flushleft}
		À Comissão de Seleção de Candidatos para Doutorado em Estatística\\
		Departamento de Estatística\\
		Instituto de Matemática e Estatística\\
		Universidade de São Paulo\\
		São Paulo, São Paulo
	\end{flushleft}

	Prezados membros da Comissão, \\
	
	%Inicio agradecendo a oportunidade de ter a minha candidatura apreciada no presente edital. Sou bacharel em Engenharia Ambiental e Sanitária e mestrando em Engenharia Ambiental pela Universidade Federal de Lavras (UFLA). A motivação em realizar o doutoramento em estatística vem do meu anseio em expandir o meu conhecimento em estatística, visto que a minha área de pesquisa, ciências atmosféricas, é bastante dependente de análises estatísticas robustas.
	
	%\section*{Graduação}
	
	%Em 2014, iniciei a minha graduação em Engenharia Ambiental e Sanitária sob a grade \href{https://sig.ufla.br/modulos/publico/matrizes_curriculares/index.php}{G019 - 2013/2}, há apenas a disciplina de Estatística Básica (GES101) no 3º módulo do curso. Nela são contemplados tópicos de estatística descritiva, probabilidade e suas distribuições, amostragem, correlação e regressão, entre outros. Após essa disciplina, tópicos da estatística são revistos em apenas duas disciplinas, uma no 6º módulo (GRH103	- Hidrologia I) e outra no 9º módulo (GAM114 -	Modelagem de Processos Ambientais).
	
	%O Trabalho de Conclusão de Curso, conduzido com a orientação da \href{http://lattes.cnpq.br/3943657653311716}{Prof. Dra. Sílvia de Nazaré Monteiro Yanagi}, teve como objeto de estudo a interação de métodos de estimativa de evapotranspiração potencial (Penman-Monteith – padrão FAO, Hargreaves \& Samani, Makkink, e Thornthwaite) na metodologia de balanço hídrico climatológico (BHC – Thornthwaite \& Matter). Foi desenvolvido através de dados meteorológicos obtido através do Instituto Nacional de Meteorologia (INMET), processamento em planilhas eletrônicas e uso de linguagem de programação R. Em termos estatísticos foram usadas estatística descritiva, índices de Willmott (acurácia dos modelos) e de desempenho (Camargos e Sentelhas – acurácia e precisão dos métodos), e regressão linear. O trabalho, que ainda não foi publicado, está anexado na documentação da candidatura. Em minha colação de grau fui agraciado com o 1º lugar no \href{https://ufla.br/noticias/ensino/14064-etapa-concluida-mais-de-500-novos-profissionais-formados-pela-ufla}{Prêmio de Mérito Acadêmico em Engenharia Ambiental - 2020}. 
	
	%\section*{Mestrado}
	%O ingresso ao mestrado ocorreu em novembro de 2020 sob orientação do \href{http://lattes.cnpq.br/5059318976988668}{Prof. Dr. Marcelo Vieira-Filho}. A pesquisa foi voltada a poluição do ar, em especial, a registrada nas capitais da região Sudeste do Brasil. Os dados utilizados foram obtidos através dos portais das entidades governamentais estaduais, sendo processados em linguagem R. Em termo estatístico, foram realizados a descrição dos dados, bem como testes de normalidade (Anderson-Darling) e de tendências (Mann-Kendall, Sen’s Slope e Cox-Stuart). Os resultados parciais foram publicados em congressos a nível local e nacional, enquanto os resultados finais foram submetidos em dois artigos: ..... 
	
	%Em relação as disciplinas, como obrigatória, a PEA507 – Tratamento Estatístico de Dados Ambientais trouxe uma revisão de conceitos estatísticos (descritiva, testes estatísticos, regressão, correlação e modelos lineares, e métodos numéricos) que foram aplicados na elaboração de um artigo. De eletivas, cursei PEX519 – Séries temporais, PEX518 – Regressão, e PEX820 – Data manipulation and vizualization. Em especial, a PEX – ST foi importante para introdução da aplicação de modelos SARIMA
	
	%\section*{Atividades extracurriculares}
	%Em termo de atividades extracurriculares, desenvolvi, por quase quatro anos (03/2016 a 09/2019), pesquisas na área de tratamentos de água e esgoto através de análises laboratoriais de amostras da ETA e ETE do campus sede da UFLA. Durante esse período foram publicados vários resumos simples e expandidos em congressos a nível local e nacional, onde a estatística descritiva foi amplamente utilizada.
	
	%No começo de 2019 alterei a minha área de pesquisa para as ciências atmosféricas e, como catalisador, ingressei no Núcleo de Estudos em Poluição Urbana e Agroindustrial (NEP UAI) – coordenado pelo Prof. Dr. Marcelo Vieira-Filho. A minha contribuição estendeu até 03/2022, quando já estava no mestrado. Em termos de gestão, fui conselheiro de comunicação, geral e de projetos, onde desenvolvi habilidades comunicacionais e de liderança. No campo das pesquisas, forma contempladas a poluição sonora no campus sede da UFLA, o impacto das medidas de lockdown no início da pandemia da COVID-19 (com coautoria de um artigo internacional e participação em congresso internacional), tendência temporal da concentração de poluentes do ar na Região Metropolitana de Belo Horizonte (publicação de resumo expandido em congresso internacional), e anomalias de precipitação e temperatura em Lavras, MG.
	
	Meu nome é Arthur Boari, sou natural de Lavras - MG, bacharel em Engenharia Ambiental e Sanitária e mestre em Engenharia Ambiental pela Universidade Federal de Lavras (UFLA). Escrevo para expressar meu interesse em realizar meu doutorado em estatística pela Universidade de São Paulo (USP).
	
	Iniciei minha graduação em Engenharia Ambiental e Sanitária em 2014 e a concluí em 2020, recebendo o \href{https://ufla.br/noticias/ensino/14064-etapa-concluida-mais-de-500-novos-profissionais-formados-pela-ufla}{Prêmio de Mérito Acadêmico em Engenharia Ambiental - 2020}. Acredito que minhas experiências extracurriculares sejam um diferencial em minha formação. Do terceiro ao décimo período, desenvolvi pesquisa na área de tratamento de água e esgoto, trabalhando principalmente com análises físico-químicas e biológicas, o que resultou em diversas publicações em congressos locais e nacionais. Durante esse período, fui bolsista voluntário de iniciação científica e bolsista de aprendizado técnico.
	
	No início de 2019, decidi alterar minha área de pesquisa e optei por estudar processos atmosféricos, como poluição sonora e do ar. Para auxiliar nessa empreitada, prestei processo seletivo do \href{https://sites.google.com/ufla.br/nepuai?pli=1}{Núcleo de Estudos em Poluição Urbana e Agroindustrial (NEP UAI)}, coordenado pelo \href{http://lattes.cnpq.br/5059318976988668}{Prof. Dr. Marcelo Vieira-Filho}. Minha contribuição se estendeu até março de 2022, quando já estava no mestrado. Em termos de gestão, fui conselheiro de comunicação, geral e de projetos, desenvolvendo habilidades comunicacionais e de liderança. No campo das pesquisas, foram contempladas a poluição sonora no \textit{campus} sede da UFLA (resultando na publicação de um \href{https://sites.google.com/ufla.br/nepuai/publica%C3%A7%C3%B5es/livros?authuser=0}{livro}), o impacto das medidas de lockdown no início da pandemia da COVID-19 (com coautoria de um \href{https://link.springer.com/article/10.1007/s11869-020-00959-8}{artigo internacional} e participação em \href{https://www.inicepg.univap.br/cd/INIC_2021/anais/arquivos/RE_0771_0575_01.pdf}{congresso internacional} ), a tendência temporal da concentração de poluentes do ar na Região Metropolitana de Belo Horizonte (publicação de resumo expandido em congresso internacional) e anomalias de precipitação e temperatura em Lavras, MG. Através do NEP UAI, conheci a linguagem R de programação e desde então tenho desenvolvido habilidades que envolvem a produção de mapas e gráficos (em especial, o pacote \textit{ggplot2}), dashboards (\textit{flexdashboard}), produção de documentos em \textit{rmarkdown} e a construção de um portfólio no \href{https://arthurboari.github.io/arthurboari/}{\textit{GitHub}}.
	
	Ainda durante a minha graduação, eu desenvolvi o meu Trabalho de Conclusão de Curso sob a orientação da \href{http://lattes.cnpq.br/3943657653311716}{Prof. Dra. Sílvia de Nazaré Monteiro Yanagi}. Nesse trabalho, eu estudei a interação de métodos de estimativa de evapotranspiração potencial (Penman-Monteith - padrão FAO, Hargreaves \& Samani, Makkink e Thornthwaite) na metodologia de balanço hídrico climatológico (BHC, metodologia de Thornthwaite \& Matter). Utilizei dados meteorológicos obtidos através do Instituto Nacional de Meteorologia (INMET), processamento em planilhas eletrônicas e linguagem de programação R. Em termos estatísticos, utilizei estatística descritiva, índices de Willmott (para avaliar a acurácia dos modelos) e de desempenho (Camargos e Sentelhas, para avaliar a acurácia e precisão dos métodos) e regressão linear. O trabalho, que ainda não foi publicado, está anexado na documentação da minha candidatura.
	
	Minha pós-graduação em nível de mestrado em Engenharia Ambiental teve início em novembro de 2020 e foi concluída em fevereiro de 2023 sob orientação do \href{http://lattes.cnpq.br/5059318976988668}{Prof. Dr. Marcelo Vieira-Filho}. Meu estudo foi focado na poluição do ar, mais especificamente na registrada nas capitais da região Sudeste do Brasil. Utilizei dados obtidos através dos portais das entidades governamentais estaduais e os processei utilizando a linguagem R. A motivação do estudo foi verificar a tendência da concentração de material particulado (MP$_{2.5}$ e MP$_{10}$) e ozônio (O$_3$) nessas capitais, e para isso, usei testes de tendências (Mann-Kendall, Sen's Slope e Cox-Stuart). Apresentei os resultados parciais em congressos locais e \href{http://www.meioambientepocos.com.br/ANAIS2022/76%20-%20244016_crescimento-da-concentrao-de-materiais-particulados-e-oznio-em-capitais-brasileiras.pdf}{nacional}, enquanto os resultados finais foram publicados em dois artigos: "Air pollution trends and exceedances: ozone and particulate matter outlook in Brazilian highly urbanized zones" (submetido à Environmental Science \& Pollution Research) e "Tendências nas concentrações de poluentes atmosféricos no período de 2015-2019: caracterização a partir do MAPBIOMAS em centros urbanos do Sudeste do Brasil" (em elaboração). Destaquei, em particular, as tendências de aumento na concentração de MP$_{2.5}$ e O$_3$ para São Paulo, além da tendência de aumento das ultrapassagens dos padrões internacionais.
	
	Quanto às disciplinas, como obrigatória, cursei PEA507 - Tratamento Estatístico de Dados Ambientais, que me trouxe uma revisão de conceitos estatísticos (descritiva, testes estatísticos, regressão, correlação e modelos lineares, e métodos numéricos) que foram aplicados na elaboração de um artigo. Já entre as disciplinas eletivas, selecionei PEX519 - Séries Temporais (modelos ARMA, ARIMA, SARIMA, GARCH, dentre outros), PEX518 - Regressão (regressão linear simples e multivariada) e PEX820 - Data Manipulation and Visualization. A escolha dessas disciplinas foi motivada pela necessidade de aprofundamento e orientação em relação ao projeto do mestrado. O grande volume de equações complexas da disciplina PEX519 me motivou a aprender a escrever documentos em \LaTeX, e desde então tenho aprimorado essa habilidade.

	
	A minha motivação em pleitear uma vaga no processo seletivo do doutorado em estatística vem principalmente da minha vontade de aprofundar o meu conhecimento em métodos estatísticos avançados e aplicá-los na minha área de formação, a engenharia ambiental. A estatística é uma ferramenta crucial na análise de dados ambientais, e acredito que um doutorado nessa área pode me fornecer as habilidades e o conhecimento necessários para trabalhar em projetos mais complexos e desafiadores.
	
	Além disso, a estatística tem uma vasta gama de aplicações na área de poluição do ar. Por exemplo, pode ser usada na modelagem da dispersão de poluentes atmosféricos, no cálculo de emissões de poluentes por fontes específicas e na avaliação de impactos ambientais. Também pode ser utilizada na análise de séries temporais de dados de poluição do ar, permitindo identificar tendências e padrões de variação que podem ser utilizados para orientar ações de controle e prevenção da poluição.
	
	Outra aplicação importante da estatística na minha área é a estimação de tendências com séries incompletas de dados. Muitas vezes, os dados ambientais são coletados em intervalos irregulares e podem apresentar lacunas ou falhas na coleta, o que torna a análise desses dados ainda mais desafiadora. Nesse sentido, técnicas estatísticas avançadas, como a imputação de dados faltantes e a análise de séries temporais com dados incompletos, podem ser úteis na obtenção de estimativas confiáveis das tendências de poluição do ar ao longo do tempo.
	
	Em resumo, um doutorado em estatística pode oferecer as ferramentas e habilidades necessárias para aprimorar a análise de dados ambientais e fornecer insights valiosos para a tomada de decisões em projetos de controle e prevenção da poluição do ar. Estou entusiasmado com a possibilidade de mergulhar nessa área de pesquisa e contribuir para o avanço do conhecimento em estatística aplicada ao meio ambiente.
	
	\label{LastPage}
\end{document}